\section{Testy}

Testowanie działania aplikacji zostało przeprowadzone dla dwóch wartości progu podobieństwa: 0,85 i 0,80. Druga wartość została sprawdzonam, by upewnić się, że próg 0,85 nie jest zbyt wysoki. Wyniki testów przedstawiono w tabelach 2. i 3. Kolumny \textit{True positive} i \textit{False positive} oznaczają kolejno: prawidłowe rozpoznanie słowa oraz brak detekcji lub wykrycie niewłaściwego wyrazu.\\
W tabeli 4. przedstawiono średnie procentowe wartości rozpoznań w stosunku do słów, które powinny zostać znalezione.


\begin{table}
	\centering
	\caption{Wyniki dla progu podobieństwa 0,85}
	\resizebox{\columnwidth}{!}{%
	\begin{tabular}{ | c | c | c | c | c | }
		\hline
		\textbf{L.p.} & \textbf{Nazwa nagrania} & \textbf{Słowa do wykrycia} & \textbf{True positive} & \textbf{False positive} \\
		\hline
		1. & wszystko\_JP & 4 & 1 & 0  \\ \hline
		2. & wszystko\_JP2 & 4 & 2 & 1  \\ \hline
		3. & fotel\_x3\_JP & 3 & 1 & 2  \\ \hline
		4. & ksiazka\_x3\_JP & 3 & 1 & 0  \\ \hline
		5. & krzeslo\_x3\_JP & 3 & 1 & 0  \\ \hline
		6. & ksiazka\_x5\_MK\_laud & 5 & 5 & 0 \\ \hline
		7. & ksiazka\_x5\_MK & 5 & 1 & 2 \\ \hline
		8. & ksiazka\_x4\_2\_MK & 4 & 2 & 0 \\ \hline
		9. & ksiazka\_x4\_fotel\_x1\_MK\_laud & 5 & 5 & 0  \\ \hline
		10. & ksiazka\_krzeslo\_fotel\_MK\_3 & 3 & 0 & 2  \\ \hline
		11. & ksiazka\_krzeslo\_fotel\_MK\_2 & 3 & 1 & 3  \\ \hline
		12. & ksiazka\_krzeslo\_fotel\_MK & 3 & 0 & 4  \\ \hline
	\end{tabular}
}
\end{table}

\begin{table}
	\centering
	\caption{Wyniki dla progu podobieństwa 0,8}
	\resizebox{\columnwidth}{!}{%
		\begin{tabular}{ | c | c | c | c | c | }
			\hline
			\textbf{L.p.} & \textbf{Nazwa nagrania} & \textbf{Słowa do wykrycia} & \textbf{True positive} & \textbf{False positive} \\
			\hline
			1. & wszystko\_JP & 4 & 2 & 3  \\ \hline
			2. & wszystko\_JP2 & 4 & 3 & 4  \\ \hline
			3. & fotel\_x3\_JP & 3 & 1 & 3  \\ \hline
			4. & ksiazka\_x3\_JP & 3 & 1 & 0  \\ \hline
			5. & krzeslo\_x3\_JP & 3 & 1 & 1  \\ \hline
			6. & ksiazka\_x5\_MK\_laud & 5 & 5 & 0  \\ \hline
			7. & ksiazka\_x5\_MK & 5 & 1 & 5  \\ \hline
			8. & ksiazka\_x4\_2\_MK & 4 & 1 & 2 \\ \hline
			9. & ksiazka\_x4\_fotel\_x1\_MK\_laud & 5 & 5 & 2  \\ \hline
			10. & ksiazka\_krzeslo\_fotel\_MK\_3 & 3 & 3 & 4  \\ \hline
			11. & ksiazka\_krzeslo\_fotel\_MK\_2 & 3 & 1 & 5 \\ \hline
			12. & ksiazka\_krzeslo\_fotel\_MK & 3 & 2 & 9  \\ \hline
		\end{tabular}
	}
\end{table}

\begin{table}
	\centering
	\caption{Średnie procentowe wartości rozpoznań względem ilości szukanych słów}
	\resizebox{\columnwidth}{!}{%
		\begin{tabular}{ | c | c | c | }
			\hline
			\textbf{Próg podobieństwa} & \textbf{True positive [\%]} & \textbf{False positive [\%]}  \\
			\hline
			0,85 & 39,86 & 35,97  \\ \hline
			0,80 & 55,83 & 91,52  \\ \hline
		\end{tabular}
	}
\end{table}
