\section{Wnioski}

Na podstawie uzyskanych wyników, 
rzeczy które można ulepszyć:

\subsubsection{Podział nagrania na słowa}
Automatyczny podział nagrania na słowa, w przeciwieństwie do zastosowanego podziału ręcznego pozwoliłby uniknąć różnic w bazie wzorców spowodowanych niedokładnością określenia początku i końca słowa. Przyspieszyłby proces i łatwość tworzenia bazy wzorców, co pozwoliłoby na stworzenie większej bazy a zatem przeprowadzenia większej ilości badań.

\subsubsection{Wyrównanie poziomu dźwięku}
Mimo analizy w dźwięku w dziedzinie częstotliwości, róźnice w nagraniach spowodowane różną siłą głosu, odległością od mikrofonu okazały się wystarczające aby przeprowadzić skuteczne rozpoznawanie. Częściowo problemy te powinno rozwiązać wyrównanie poziomu analizowanego dźwięku, jak również oczyszczenie go z odgłosów tła, tab aby analizować sam głos.

\subsubsection{Klasyfikator}
Użycie innego klasyfikatora

\subsubsection{Długość trwania wzorca}
Opracowanie algorytmu oraz klasyfikatora pozwalającego znajdować podobieństwo pomiędzy nagraniami tego samego słowa, ale różnych czasach trwania. 