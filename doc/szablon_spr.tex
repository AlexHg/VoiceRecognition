%%%%%%%%%%%%%%%%%%%%%%%%%%%%%%%%%%%%%%%%%
% University/School Laboratory Report
% LaTeX Template
% Version 3.1 (25/3/14)
%
% This template has been downloaded from:
% http://www.LaTeXTemplates.com
%
% Original author:
% Linux and Unix Users Group at Virginia Tech Wiki 
% (https://vtluug.org/wiki/Example_LaTeX_chem_lab_report)
%
% License:
% CC BY-NC-SA 3.0 (http://creativecommons.org/licenses/by-nc-sa/3.0/)
%
%%%%%%%%%%%%%%%%%%%%%%%%%%%%%%%%%%%%%%%%%

%----------------------------------------------------------------------------------------
%	PACKAGES AND DOCUMENT CONFIGURATIONS
%----------------------------------------------------------------------------------------

\documentclass{article}

\usepackage{siunitx} % Provides the \SI{}{} and \si{} command for typesetting SI units
\usepackage{graphicx} % Required for the inclusion of images
%\usepackage{natbib} % Required to change bibliography style to APA
\usepackage{amsmath} % Required for some math elements 
\usepackage[margin=1.5in]{geometry}


\setlength\parindent{0pt} % Removes all indentation from paragraphs

\renewcommand{\labelenumi}{\alph{enumi}.} % Make numbering in the enumerate environment by letter rather than number (e.g. section 6)

%ustawienie jezyka polskiego
\usepackage{polski}
\usepackage[utf8]{inputenc}
\usepackage[T1]{fontenc}
\usepackage{multirow}
\usepackage{rotating}
\usepackage{float}
\usepackage[table]{xcolor}
\usepackage{enumerate}
\usepackage{subcaption}


\graphicspath{ {images/} }


%----------------------------------------------------------------------------------------
%	DOCUMENT INFORMATION
%----------------------------------------------------------------------------------------

\title{Projektowanie systemów z dostępem w języku naturalnym\\
	\vspace{5mm}
	\textbf{Opracowanie aplikacji umożliwiającej w „mówionym” tekście (nagraniu) wykrycie
		określonych słów}
}

\author{\\
	\\\textbf{Autorzy:}
	\\Maciej Kiedrowski, nr indeksu: 200105
	\\Joanna Piątek, nr indeksu: 199966
	\\\\
	\\
	\\\textbf{Grupa:} Czwartek TN/11:15}
\date{\textbf{Data oddania:} 26.01.2017}

\begin{document}

\maketitle % Insert the title, author and date

\begin{center}
\begin{tabular}{l r}
%Data wykonania ćwiczenia: & ... \\ % Date the experiment was performed
%Partners: & James Smith \\ % Partner names
%& Mary Smith \\
Prowadzący: & dr inż. Dariusz Banasiak 

\end{tabular}
\end{center}


\newpage
\tableofcontents 	%spis tresci
\newpage
Szukanie słowa w nagraniu:
\\
\begin{enumerate}
\item 
{
	Za pomocą korelacji fragmentu i nagrania:
	\begin{itemize}
		\item Bezbłędne rozpoznanie identycznego fragmentu w nagraniu
		\item Rozpoznanie danego słowa wypowiedzianego przez tę samą osobę w innym nagraniu
		\item Nie udało się rozpoznać słowa w nagraniu głosu innej osoby
	\end{itemize}
}

\item
{
	10-milisekundowe fragmenty nagrania samego słowa oraz tego, w którym szukane jest słowo poddano 10-punktowej FFT. Następnie została zastosowana korelacja macierzy FFT dla par: 
	\begin{itemize}
		\item Macierz FFT nagranego słowa (dla każdego framentu wektor FFT)
		\item Macierz FFT nagrania, w którym szukamy (j.w.)
	\end{itemize}
}
\end{enumerate}
 
% If you wish to include an abstract, uncomment the lines below
% \begin{abstract}
% Abstract text
% \end{abstract}

%----------------------------------------------------------------------------


%\begin{figure}[h]
%	\begin{center}
%		\includegraphics[width=0.65\textwidth]{placeholder} % Include the image placeholder.png
%		\caption{Figure caption.}
%	\end
%	{center}
%\end{figure}


\end{document}